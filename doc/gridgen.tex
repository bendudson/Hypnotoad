%% Manual for grid generator

\documentclass[12pt, a4paper]{article}
\usepackage[nofoot]{geometry}
\usepackage{graphicx}
\usepackage{fancyhdr}
\usepackage{amsfonts}

%% Modify margins
\addtolength{\oddsidemargin}{-.25in}
\addtolength{\evensidemargin}{-.25in}
\addtolength{\textwidth}{0.5in}
\addtolength{\textheight}{0.25in}
%% SET HEADERS AND FOOTERS

\pagestyle{fancy}
\fancyfoot{}
\renewcommand{\sectionmark}[1]{         % Lower case Section marker style
  \markright{\thesection.\ #1}}
\fancyhead[LE,RO]{\bfseries\thepage}    % Page number (boldface) in left on even
                                        % pages and right on odd pages 
\renewcommand{\headrulewidth}{0.3pt}

\newcommand{\code}[1]{\texttt{#1}}
\newcommand{\file}[1]{\texttt{\bf #1}}

%% commands for boxes with important notes
\newlength{\notewidth}
\addtolength{\notewidth}{\textwidth}
\addtolength{\notewidth}{-3.\parindent}
\newcommand{\note}[1]{
\fbox{
\begin{minipage}{\notewidth}
{\bf NOTE}: #1
\end{minipage}
}}

\newcommand{\pow}{\ensuremath{\wedge} }
\newcommand{\poweq}{\ensuremath{\wedge =} }

\newcommand{\deriv}[2]{\ensuremath{\frac{\partial #1}{\partial #2}}}
\newcommand{\dderiv}[2]{\ensuremath{\frac{\partial^2 #1}{\partial {#2}^2}}}
\newcommand{\Vpar}{\ensuremath{V_{||}}}
\newcommand{\Gradpar}{\ensuremath{\partial_{||}}}
\newcommand{\Divpar}{\ensuremath{\nabla_{||}}}
\newcommand{\DivXgradX}[2]{\ensuremath{\nabla_\psi\left(#1\partial_\psi #2\right)}}
\newcommand{\DivParGradPar}[2]{\ensuremath{\nabla_{||}\left(#1\partial_{||} #2\right)}}

\newcommand{\apar}{\ensuremath{A_{||}}}
\newcommand{\hthe}{\ensuremath{h_\theta}}
\newcommand{\Bp}{\ensuremath{B_\theta}}
\newcommand{\Bt}{\ensuremath{B_\zeta}}

\newcommand{\Vec}[1]{\ensuremath{\mathbf{#1}}}
\newcommand{\bvec}{\Vec{b}}
\newcommand{\kvec}{\Vec{\kappa}}
\newcommand{\vvec}{\Vec{v}}
\newcommand{\bxk}{\bvec_0\times\kvec_0\cdot\nabla}
\newcommand{\Bvec}{\Vec{B}}
\newcommand{\Bbar}{\overline{B}}
\newcommand{\Lbar}{\overline{L}}
\newcommand{\Tbar}{\overline{T}}
\newcommand{\Jvec}{\Vec{J}}
\newcommand{\Jpar}{J_{||}}
\newcommand{\delp}{\nabla_\perp^2}
\newcommand{\Div}[1]{\ensuremath{\nabla\cdot #1 }}
\newcommand{\Curl}[1]{\ensuremath{\nabla\times #1 }}
\newcommand{\rbp}{\ensuremath{R\Bp}}
\newcommand{\rbpsq}{\ensuremath{\left(\rbp\right)^2}}

\begin{document}

\title{Tokamak grid generator in IDL}
\author{B.Dudson\dag, M.V.Umansky\ddag \\
\dag University of York \\
\ddag Lawrence Livermore National Laboratory}

\maketitle

\tableofcontents

\section{Using the grid generator}

The function \code{create\_grid} takes a 2D array of $\psi$ values, and 
produces an orthogonal mesh aligned with the flux-surfaces.

Settings to control the resulting mesh are:
\begin{itemize}
\item \code{psi\_inner}, the normalised $\psi$ of the innermost flux surface. 
  This can be either a scalar or an array:
  \begin{itemize}
    \item \code{{\bf scalar}}: This value is used for the core and all PF regions
    \item \code{{\bf array[0]}}: The inner normalised $\psi$ for the core
    \item \code{{\bf array[1..n\_xpoint]}}: Inner $\psi$ to use for each PF region
      (see section~\ref{sec:numbering})
  \end{itemize}
\item \code{psi\_outer}, normalised $\psi$ of outermost surface. Can also be
  either a scalar or array:
  \begin{itemize}
  \item \code{{\bf scalar}}: This value is used for the core and all PF regions
  \item \code{{\bf array[0..(n\_xpoint-1)]}}: Outer normalised $\psi$ for each SOL region
    (one per x-point)
  \end{itemize}
\item \code{nrad} Number of radial grid points
  \begin{itemize}
  \item \code{{\bf scalar}}: Total number of radial grid points. Automatically divides this
    between regions.
  \item \code{{\bf array[0]}}: Number of radial grid points in the core
  \item \code{{\bf array[1..(n\_xpoint-1)]}}: Radial grid points between separatrices
    (going outwards from core to edge)
  \item \code{{\bf array[n\_xpoint]}}: Radial grid points outside last separatrix
  \end{itemize}
\item \code{npol} Number of poloidal grid points.
  \begin{itemize}
  \item \code{{\bf scalar}}: Total number of points. Divides between regions based on
    poloidal arc lengths
  \item \code{{\bf array[0..(3*n\_xpoint-1)]}}: Number of points in each poloidal region. 
    See section~\ref{sec:numbering} for numbering scheme.
  \end{itemize}
\item \code{rad\_peaking}
\item \code{pol\_peaking}
\end{itemize}

\section{DCT}

DCT of 2D NxM $f\left(x,y\right)$
\[
F\left(u, v\right) = \sqrt{\frac{2}{N}}\sqrt{\frac{2}{M}}\Lambda\left(u\right)\Lambda\left(v\right)\sum_{i=0}^{N-1}\sum_{j=0}^{M-1} f\left(i, j\right) \cos\left[\frac{\pi u}{2N}\left(2i+1\right)\right]\cos\left[\frac{\pi v}{2M}\left(2j+1\right)\right]
\]
where $\Lambda\left(i\right) = 1/\sqrt{2}$ for $i=0$, and $\Lambda\left(i\right) = 1$ otherwise
 
\section{Finding critical points}

To find x- and o-points, 

\section{Region numbering}
\label{sec:numbering}


\section{Separatrices}

Having found the x-point locations, the separatrices need to be found.
First step is to calculate the lines going through the x-point:

Close to an x-point, approximate the change in $\psi$ by
\[
\delta\psi = \frac{1}{2}\psi_{xx} x^2 + \frac{1}{2}\psi_{yy}y^2 + \psi_{xy} xy
\]

The two lines through the x-point are then given by where this is zero:

\[
\frac{1}{2}\psi_{yy}y^2 + \psi_{xy}xy + \frac{1}{2}\psi_{xx} x^2 = 0
\]
Which has the solution
\[
y = \frac{ -\psi_{xy}x \pm \sqrt{\psi_{xy}^2x^2 - \psi_{yy}\psi_{xx}x^2}}{\psi_{yy}}
\]
i.e.
\[
y = \frac{1}{\psi_{yy}}\left(-\psi_{xy} \pm \sqrt{\psi_{xy}^2 - \psi_{yy}\psi_{xx}}\right)x
\]
Note that if $\psi_{yy} = 0$ then the solutions are $x = 0$ and $y = -\frac{\psi_{xx}}{2\psi_{xy}}x$

\end{document}
